% ELEMENTOS PRÉ-TEXTUAIS
\makeoddhead{chapfirst}{}{}{}
% ----------------------------------------------------------
% Capa
% ----------------------------------------------------------
 \ifthenelse{\equal{\terCapa}{Sim}}{
\imprimircapa}{}

% Folha de rosto
% ----------------------------------------------------------
\imprimirfolhaderosto*

% Inserir a ficha bibliografica
% ----------------------------------------------------------
 \ifthenelse{\equal{\terFichaCatalografica}{Sim}}
 {\insereFichaCatalografica{}\cleardoublepage}
 {}

% Inserir errata
% ----------------------------------------------------------
 \ifthenelse{\equal{\terErrata}{Sim}}
 {\begin{errata}%\color{blue}
   \imprimirerrata
  \end{errata}}
 {}

% Inserir folha de aprovação
% ----------------------------------------------------------
\ifthenelse{\equal{\terTermoAprovacao}{Sim}}{
\insereAprovacao}{}

% Dedicatória
% ----------------------------------------------------------
\ifthenelse{\equal{\terDedicatoria}{Sim}}{
\begin{dedicatoria}
   \vspace*{\fill}
   \centering
   \noindent
   \DedicatoriaTexto
   \vspace*{\fill}
\end{dedicatoria}
}{}

% Agradecimentos
% ----------------------------------------------------------

 \ifthenelse{\equal{\terAgradecimentos}{Sim}}
 {\begin{agradecimentos}
    \AgradecimentosTexto
  \end{agradecimentos}
  }{}
% Epígrafe
% ----------------------------------------------------------

\ifthenelse{\equal{\terEpigrafe}{Sim}}{
\begin{epigrafe}
    \vspace*{\fill}
	\begin{flushright}
        \EpigrafeTexto
	\end{flushright}
\end{epigrafe}
}{}

% RESUMOS
% ----------------------------------------------------------
% resumo em português
%\setlength{\absparsep}{18pt} % ajusta o espaçamento dos parágrafos do resumo
 \ifthenelse{\equal{\terResumos}{Sim}}{
\begin{resumo}
    \ResumoTexto
    
    %\vspace{\onelineskip}
    \noindent 
    \textbf{Palavras-chaves}: \PalavraschaveTexto
\end{resumo}

%% resumo em inglês
\begin{resumo}[ABSTRACT]
 \begin{otherlanguage*}{english}
   \AbstractTexto
   
   %\vspace{\onelineskip}
   \noindent 
   \textbf{Key-words}: \KeywordsTexto
 \end{otherlanguage*}
\end{resumo}


% resumo em francês 
\ifthenelse{\equal{\Resume}{}}
{}
{
 \begin{resumo}[RESUME]%Résumé
  \begin{otherlanguage*}{french}
     \Resume
     
     %\vspace{\onelineskip}
     \noindent      
     \textbf{Mots clés}: \Motscles
  \end{otherlanguage*}
 \end{resumo}
} 

% resumo em espanhol
\ifthenelse{\equal{\Resume}{}}{}
{ \begin{resumo}[RESUMEN]
  \begin{otherlanguage*}{spanish}
    \Resumen 
   
   %\vspace{\onelineskip}
   \noindent    
    \textbf{Palabras clave}: \Palabrasclave
  \end{otherlanguage*}
 \end{resumo}
}
}{}

% inserir lista de ilustrações
% ----------------------------------------------------------
\ifthenelse{\equal{\terListaFiguras}{Sim}}{
%\pdfbookmark[0]{\listfigurename}{lof}
\listoffigures*
\cleardoublepage
}{}

% inserir lista de quadros
% ----------------------------------------------------------
\ifthenelse{\equal{\terListaQuadros}{Nao}}{
%\pdfbookmark[0]{\listtablename}{lot}
\listofquadros*
\cleardoublepage
}{}

% inserir lista de tabelas
% ----------------------------------------------------------
\ifthenelse{\equal{\terListaTabelas}{Sim}}{
%\pdfbookmark[0]{\listtablename}{lot}
\listoftables*
\cleardoublepage
}{}


% inserir lista de abreviaturas e siglas 
% inserir lista de símbolos
% ----------------------------------------------------------

\imprimirlistadesiglas
\cleardoublepage

 % \ifthenelse{\equal{\terSiglasAbrev}{Sim}}{
    % \imprimirlistadesiglas
    % \cleardoublepage
    % \imprimirlistadesimbolos
    % \cleardoublepage
 % }{}

% inserir o sumario
\makeoddhead{chapfirst}{}{}{}
% ----------------------------------------------------------
\ifthenelse{\equal{\terSumario}{Sim}}{
%\pdfbookmark[0]{\contentsname}{toc}
\pagestyle{empty}
\tableofcontents*
\thispagestyle{empty}
%\cleardoublepage
}{}
 

 
 
