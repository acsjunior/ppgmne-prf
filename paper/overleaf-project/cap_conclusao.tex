\chapter{Considerações Finais}

Uma vez que a problemática da distribuição eficiente dos postos de fiscalização (UOP) da Polícia Rodoviária Federal (PRF) afeta diretamente o consumo de recursos públicos e a qualidade do serviço prestado pela PRF à população, esta dissertação buscou, com uma abordagem baseada no problema de p-Medianas, determinar o número e a localização das UOPs no estado do Paraná, com o objetivo de minimizar as distâncias entre as UOPs e os locais com registros de acidentes.

No primeiro cenário, dentre os quatro avaliados durante o estudo, o objetivo foi encontrar o número e a localização das UOPs entre todas as possíveis localizações mapeadas. Nesta etapa foi concluído que 39 seria o número ideal de UOPs e que 5 das UOPs já existentes poderiam ser aproveitadas na rede. Além disso, foi definida uma sequência de expansão da rede até 50 UOPs com o objetivo de minimizar o impacto financeiro e operacional quando houver necessidade de adicionar novos postos. Em termos práticos esta abordagem prescreve a desativação de 28 UOPs e a construção de outras 34 unidades, mas a solução proposta não foi considerada no estudo como viável, dado o alto investimento necessário e o risco do comportamento dos acidentes no período analisado não se manter no futuro. 

Já no segundo cenário, considerando o número de UOPs definido no cenário anterior, o objetivo foi encontrar o número máximo de UOPs já existentes com o mínimo impacto na função objetivo (FO). Nesta etapa foi concluído que, com um custo de 1,15\% na FO do primeiro cenário, é possível fixar 15 das 33 UOPs já existentes na solução. Assim como no primeiro cenário, também foi definida uma sequência para expansão da rede até 50 UOPs. Em termos práticos esta abordagem prescreve a desativação de 18 UOPs e a construção de outras 24, mas pelos mesmos motivos alegados no cenário 1, a solução proposta não foi considerada viável. Entretanto, é possível que em outra UF a abordagem resulte em uma prescrição viável.

O terceiro e quarto cenário consideraram a rede com as 33 UOPs já existentes. No cenário 3, que teve como objetivo avaliar a possibilidade de reduzir o número de UOPs, foi concluído que, com um custo de 1,92\% na FO, é possível reduzir o número de UOPs da rede para 29. Na prática esta solução além de mitigar os custos para manter em funcionamento as 4 UOPs eleitas para serem desativadas, ela permite remanejar o efetivo para outras atribuições, podendo inclusive, mediante estudo futuro, serem aproveitados em postos móveis localizados conforme a demanda. Já no cenário 4, que teve como objetivo expandir a rede atual para o número de 39 UOPs definido no primeiro cenário, obteve-se em ordem sequencial as 6 UOPs a serem construídas.

A abordagem baseada no problema de p-Medianas se demonstrou sensível ao valor de $p$ escolhido, fazendo-se necessária a análise e o ajuste da solução em determinados cenários para viabilizar a expansão da rede com o menor impacto financeiro e operacional possível. Por outro lado, a possibilidade de escolher diferentes valores de $p$ permitiu a análise de diferentes cenários, trazendo flexibilidade a esta abordagem determinística ao oferecer para o tomador de decisão um leque de possibilidades. Além disso, há o risco do modelo de p-Medianas simplificar excessivamente o problema de geolocalização e, consequentemente, gerar soluções otimizadas do ponto de vista matemático, mas que podem não ser alinhadas com as necessidades operacionais da PRF. No entanto, a abordagem permite a incorporação de restrições específicas para alinhar as soluções com as necessidades operacionais da PRF.

Cinco próximos passos foram mapeados com o intuito de aumentar a robustez dos resultados obtidos: o primeiro deles é a realização de um levantamento junto à PRF sobre possíveis regras para a localização e distância entre UOPs e incluí-las como restrições nos modelos. O segundo passo é considerar nos modelos o efetivo de servidores por UOP, informação que também depende de um levantamento junto à PRF. O terceiro passo é substituir as distâncias calculadas por meio da fórmula de Haversine pelas distâncias reais extraídas via API do Google Maps ou algum outro serviço de geolocalização. O quarto passo é agregar os acidentes considerando coordenadas mais precisas, o que, consequentemente, aumentaria o número de quadrantes. E, por fim, o quinto passo é considerar como parâmetro nos modelos os resultados de um modelo de previsão de acidentes ao invés dos dados históricos.


